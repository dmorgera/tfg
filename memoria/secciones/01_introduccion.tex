\chapter{Introducción}
\section{Contexto}

\subsection{Alastria}
\textit{Alastria} \cite{alastria} es una asociación sin ánimo de lucro que fomenta la economía digital a través del desarrollo de tecnologías de registro descentralizadas/Blockchain. 
Creada en 2017 con el nombre de Red Lyra, Alastria es un consorcio que apuesta por la democratización de la blockchain en España, y por la descentralización del uso de diversos productos y servicios.
También son los creadores de \textit{Alastria ID} \cite{alastria-id}, un modelo de identidad soberana que será una de las partes centrales de este proyecto. Alastria tiene en funcionamiento actualmente dos redes, que son la red-T y la red-B, que utilizan tecnologías \textit{Quorum} \cite{quorum} y \textit{Hyperledger Besu} \cite{besu}, respectivamente. Está en desarrolló una tercera red, la red-H, pensada para el funcionamiento con Hyperledger Fabric.

\subsection{Identidad soberana}
Para entender como funciona Alastria ID, hay que entender primero que es un modelo de identidad soberana \cite{ssi}\cite{multi-ssi}.
Identidad soberana es un término que se utiliza para denominar al movimiento digital que defiende que un sujeto debería poder controlar y manejar sus datos y su identidad sin necesidad de una autoridad administrativa que los gestione, y su objetivo principal es lograr que la interacción en el mundo digital sea tan confiable y segura como fuera de el.

En el mundo offline, toda entidad, ya sean personas físicas o empresas, interactúa con otras entidades mediante el uso de credenciales, representadas por medios físicos (carnet de identidad, permiso de conducción, etc...), y representan datos personales o información sobre la persona en cuestión.

Con los modelos de identidad soberana, esta información se almacena de forma digital en forma de credenciales que se almacenan en una cartera digital o \textit{wallet} controlado por el usuario. De esta forma, no hay mas necesidad de servicios de terceros para almacenar y utilizar datos personales.

El concepto de identidad soberana se remonta a los orígenes del propio internet, y ha sido iterado en múltiples ocasiones por diferentes personas. Con estas ideas, se crearon diez principios básicos para asegurar la eficacia de la identidad soberana \cite{path-to-ssi}:
\begin{enumerate}
    \item \textbf{Existencia:} Los usuarios deben tener una existencia independiente.
    \item \textbf{Control:} Los usuarios deben poder controlar sus identidades.
    \item \textbf{Acceso:} Los usuarios deben tener acceso a sus propios datos.
    \item \textbf{Transparencia:} Los sistemas y los algoritmos deben ser transparentes.
    \item \textbf{Persistencia:} Las identidades deben durar mucho, o hasta que el usuario desee.
    \item \textbf{Portabilidad:} Las identidades deben ser transportables.
    \item \textbf{Interoperabilidad:} Las identidades se deben poder usar en el mayor número de servicios posible.
    \item \textbf{Consentimiento:} Los usuarios deben permitir el uso de sus identidades.
    \item \textbf{Minimización:} Solo debe compartirse lo mínimo necesario.
    \item \textbf{Protección:} Los derechos de los usuarios deben ser la prioridad.
\end{enumerate}
Ésta es solo una descripción básica del concepto de identidad soberana, pero hay múltiples modelos cuyas implementaciones pueden derivar de estos términos.

\section{Metodología}
Como se ha mencionado en apartados anteriores, este proyecto consiste de dos partes, un estudio teórico y una implementación practica.

\subsection{Estudio teórico}
En primer lugar, se realizara un estudio sobre el modelo de identidad Alastria ID, además de una pequeña investigación sobre blockchain, sus tipos y una comparativa teórica.

\subsection{Implementación práctica}
En la segunda parte del proyecto, se diseñaran los chaincodes de \textit{Hyperledger Fabric} \cite{fabric} en base a los smart contracts de Alastria ID, buscando un funcionamiento idéntico o similar, para luego implementarlos en una red Fabric. Se realizará una implementación parcial.

\section{Objetivos}
A lo largo del desarrollo del proyecto, se intentará cumplir con ciertos objetivos teóricos y prácticos:

\begin{itemize}
  \item Estudio del modelo de identidad Alastria ID.
  \item Aprendizaje sobre tecnologías Blockchain como Hyperledger Fabric.
  \item Aprendizaje sobre el lenguaje de programación Go.
  \item Implementación de smart contracts de Alastria ID en Hyperledger Fabric.
  \item Implementación del chaincode en una red Fabric.
\end{itemize}