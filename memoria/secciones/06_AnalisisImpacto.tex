\chapter{Análisis de impacto y desafíos}
En este capítulo se realizará un análisis del impacto potencial de los resultados obtenidos durante la realización del TFG, en los diferentes contextos para los que se aplique.
\section{Impacto potencial}
\subsection{Personal}
En el ámbito personal, la realización de este trabajo me ha supuesto un desafío a la hora de investigar sobre nuevas tecnologías, ya que he tenido que aprender de cero prácticamente todo con lo que he trabajado para el proyecto. También ha supuesto un reto el adaptar un modelo existente a una tecnología que no es del todo compatible con el mismo.\\
La parte mas complicada ha sido el aprendizaje de Hyperledger Fabric, ya que ha resultado ser una tecnología mucho mas compleja de lo esperado. Por otro lado, la parte mas sencilla de aprender ha sido Alastria ID, debido a que ya estaba familiarizado con el modelo y solo he tenido que profundizar en los smart contracts.
\subsection{Empresarial}
Tanto Quorum como Hyperledger Fabric están diseñadas con el aspecto empresarial en mente, y ambas tienen casos de uso que pueden aportar mucho valor a empresas de cualquier industria. La implementación de Alastria ID en una red más ampliaría el abanico de opciones que tendrían las empresas interesadas en soluciones de este tipo, pudiendo elegir la red mas apropiada a sus circunstancias.
\clearpage
\section{Objetivos de desarrollo sostenible}
Este proyecto se amolda sobre todo al objetivo número 9, sobre industria, innovación e infraestructuras. Sólo en España, un 96\% de personas utilizan teléfono móvil, y un 91\% tiene acceso a internet, por lo que la adopción del modelo Alastria ID en la sociedad actual conllevaría un gran avance en los sistemas de provisión de servicio, y en la gestión de los datos personales de cualquier individuo. Además, en el ámbito empresarial, incorporar el modelo aceleraría todo tipo de procesos de negocio, y aumentaría la confianza y la transparencia a la hora de realizar servicios, tanto en la empresa pública como en el sector privado.

También se ajusta al objetivo 13, de acción por el clima, ya que si se sustituye el uso de credenciales físicas o en papel por credenciales digitales, se reduce la contaminación y la producción de tarjetas, documentos, etc..., que son perjudiciales para el planeta.

Gracias a las credenciales digitales, el modelo podría adaptarse también al resto de objetivos. Por ejemplo, los objetivos 1 y 2, fin de la pobreza y hambre cero, se podrían cubrir con una credencial de vulnerabilidad económica, que podría acreditar a una familia para recibir ayudas del estado, o el objetivo 3, salud y bienestar, se podría cumplimentar con una credencial PCR, para acreditar un negativo ante pruebas COVID. La identidad digital proporcionada por el modelo de Alastria ID es un concepto transversal que, utilizada correctamente, puede ayudar a cumplir todos los objetivos de desarrollo sostenible.