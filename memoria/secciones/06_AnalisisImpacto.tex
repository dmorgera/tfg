\chapter{Análisis de impacto y desafíos}
En este capítulo se realizará un análisis del impacto potencial de los resultados obtenidos durante la realización del TFG, en los diferentes contextos para los que se aplique.
\section{Impacto potencial}
\subsection{Personal}
En el ámbito personal, la realización de este trabajo me ha supuesto un desafío a la hora de investigar sobre nuevas tecnologías, ya que he tenido que aprender de cero prácticamente todo con lo que he trabajado para el proyecto. También ha supuesto un reto el adaptar un modelo existente a una tecnología que no es del todo compatible con el mismo.\\
La parte mas complicada ha sido el aprendizaje de Hyperledger Fabric, ya que ha resultado ser una tecnología mucho mas compleja de lo esperado. Por otro lado, la parte mas sencilla de aprender ha sido Alastria ID, debido a que ya estaba familiarizado con el modelo y solo he tenido que profundizar en los smart contracts.
\subsection{Empresarial}
Tanto Quorum como Hyperledger Fabric están diseñadas con el aspecto empresarial en mente, y ambas tienen casos de uso que pueden aportar mucho valor a empresas de cualquier industria. La implementación de Alastria ID en una red más ampliaría el abanico de opciones que tendrían las empresas interesadas en soluciones de este tipo, pudiendo elegir la red mas apropiada a sus circunstancias.

\section{Objetivos de desarrollo sostenible}
Este apartado está referido a los ODS, especificados \href{https://www.un.org/sustainabledevelopment/es/objetivos-de-desarrollo-sostenible/}{aquí}.
