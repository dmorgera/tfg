\chapter*{Resumen}
Alastria ID es un modelo de identidad digital propuesto por el consorcio Alastria para su uso en servicios digitales, incluso más allá de la propia tecnología blockchain e inspirado en el concepto de la Identidad Digital Soberana (SSI).

Los modelos de identidad digital y las redes blockchain en general son conceptos muy innovadores y con muchas implicaciones para el desarrollo de la tecnología y la gestión de muchos aspectos de la sociedad, como la trazabilidad de intercambios de todo tipo con blockchain, o la gestión de nuestros propios datos personales de forma segura y transparente con Alastria ID.

Actualmente, los smart contracts de Alastria ID están codificados en Solidity, diseñados para una red Quorum. Hyperledger Fabric comparte varias características y casos de uso con Quorum, ya que están ambas diseñadas para el ámbito empresarial, y siendo Fabric la red privada mas popular, puede aportar mucho valor al modelo Alastria ID. Siguiendo una de las muchas lineas de investigación disponibles en este campo, se analizará la viabilidad del modelo en Hyperledger Fabric, y se realizará un estudio comparativo entre las dos tecnologías, con el objetivo de diseñar unos smart contracts para una posible implementación de Alastria ID en redes Hyperledger Fabric.

%%--------------
\newpage
%%--------------

\chapter*{Abstract}
Alastria ID is a digital identity model proposed by the Alastria consortium for use in digital services, even beyond blockchain technology itself and inspired by the concept of Self Sovereign Identity (SSI).

Digital identity models and blockchain networks in general are very innovative concepts, with many implications for the development of technology and the management of various aspects of society, such as the traceability of all types of exchanges of all kinds using blockchain, or the secure and transparent management of our identity and personal data thanks to Alastria ID.

Currently, Alastria ID's smart contracts are coded in Solidity, designed to run in Quorum blockchain networks. Hyperledger Fabric shares several features and use cases with Quorum, as they are both designed for the enterprise environment, and with Fabric being the most popular private network, it can add much value to the Alastria ID model. Following along one of many research lines in the field, the viability of Alastria ID's model in Hyperledger Fabric will be analyzed,  and a comparative study between the two technologies will be carried out, with the purpose of designing smart contracts for a possible Alastria ID implementation in Hyperledger Fabric networks.

